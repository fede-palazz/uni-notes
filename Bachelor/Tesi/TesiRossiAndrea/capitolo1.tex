\chapter{Introduzione}

\section{Obiettivo}
Lo svolgimento di questo progetto è focalizzato sullo sviluppo di una Applicazione Web che consenta di visualizzare piante che sono di interesse nell'ambito di orti botanico in rete.
Quest'ultima deve essere in grado di poter elaborare e visionare dati riguardo quest'ambito, utilizzando specifici linguaggi e software studiati durante il percorso di studio. \newline

\section{Struttura della Tesi}
La tesi viene strutturata principalmente come introduzione all'applicazione web.
Verranno prima affrontati concetti elementari per poi introdurre più nozioni su tali elementi, ogni parte è propedeutica alla successiva.
Viene quindi strutturata come segue:
\begin{itemize}
\item nella prima parte verranno introdotti gli strumenti software e i linguaggi utilizzati suddivisi lato front e back end;
\item nella seconda parte verrà spiegato quali sono state le fasi della progettazione dell'applicazione web. Si parte dal database logic, poi si analizzano gli script php utilizzati nel back end infine il restante codice del front end;
\item nella terza parte  vengono espletati tutti gli accorgimenti necessari affinché si abbia un corretto set-up dell'applicazione web; 
\item nella quarta parte verranno espletati altri accorgimenti affinché si abbia la migliore esperienza di navigazione mobile dal browser Safari. Dopodiché verrà fatta una panoramica dell'applicazione web, guidando l'utente tramite l'uso di screen;
\item nell'ultima parte verranno esposte le conclusioni e verranno introdotti gli sviluppi futuri;
\end{itemize}
\section{La scelta dell'Applicazione Web}
La scelta dell'applicazione web anziché un'applicazione software scritta in Java o C\# è motivata dal fatto che essenzialmente un'applicazione web può essere eseguita su qualunque dispositivo dotato di un web browser. Altrimenti il percorso di sviluppo doveva essere suddiviso a seconda del dispositivo, quindi utilizzando Android Studio per un dispositivo Android, Xcode per un dispositivo Apple e via dicendo. \newline
\section{Indipendenza da software esterni per la gestione del database}
Una cosa di fondamentale importanza che è stata implementata con successo in questo progetto è l'indipendenza da un modulo esterno per la gestione del database ad eccezione della creazione del database, tabelle e dati di default.
Tale scelta comporta una maggiore flessibilità ed anche facilità per la visualizzazione e manipolazione dei dati del database senza dover ogni volta accedere a un client di gestione di basi di dati o al modulo PhpMyAdmin.

\section{La scelta di impiegare due Front-end}
Sono stati realizzati due front-end che consumano lo stesso back-end. \newline
Il front-end \textbf{basic} come suggerisce il nome, è più basilare, è privo di effetti grafici e rispecchia una interfaccia web light. \newline
Il front-end \textbf{full} invece è completo anche dal punto di vista grafico. Rispecchia gli standard attuali \textit{Material Design} dell'odierna programmazione web.
\newline \newline
Concludendo si può dire che sviluppare due front end è stato un ottimo esercizio per raggiungere gli stessi obiettivi utilizzando linguaggi di programmazione, costrutti ed approcci diversi per ogni front end. 


